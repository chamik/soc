\makeatletter
\renewcommand{\@chapapp}{}% Not necessary...
\newenvironment{chapquote}[2][2em]
  {\setlength{\@tempdima}{#1}%
   \def\chapquote@author{#2}%
   \parshape 1 \@tempdima \dimexpr\textwidth-2\@tempdima\relax%
   \itshape}
  {\par\normalfont\hfill--\ \chapquote@author\hspace*{\@tempdima}\par\bigskip}
\makeatother

\chapter*{Úvod}
\addcontentsline{toc}{chapter}{Úvod} % přidá položku úvod do obsahu

Jako kdyby to bylo včera, když jsem pár let zpátky u nás na gymnáziu podstoupil takzvaný \enquote{rozřazovák}. Program, ve kterém jsme pracovali, byl zastaralý už jen od pohledu a šel triviálně rozbít -- taky si dobře vzpomínám, jak do dveří vtrhla v obličeji rudá učitelka angličtiny a začala hubovat spolužáka, \textit{proč ten test odevzdal třikrát??}

Program nakonec přestal fungovat úplně. Od té doby se poslechová část dělá na papír a zbytek na počítači, což je nepohodlné nejen pro žáky, ale hlavně pro učitele. Rozhodl jsem se ho tedy přetvořit. Ani tak z dobré vůle, jako spíše z čistě pragmatického důvodu:~špatný software mě rozčiluje; ještě o to více, když je využíván ve výukové sféře.

Mým cílem je vytvořit program, který dokáže spolehlivě otestovat desítky lidí naráz a který je intuitivní na používání. Měl by používat nejmodernější volně dostupné technologie a hlavně by měl život lidí usnadnit, ne naopak. Jak zní nechvalně známý citát:

\begin{chapquote}{George Bernard Shaw, \textit{Člověk a Nadčlověk, 1903}}
    \enquote{Kdo umí, ten umí, kdo neumí, ten učí.}
    % \enquote{He who can, does; he who cannot, teaches}
\end{chapquote}

Tudíž věřím, že spíše než v lidech je klíč k dobré výuce ve kvalitních výukových postupech a materiálech -- a to je něco, s čím my, programátoři, dokážeme pomoci; nevyužít interaktivity, kterou nám moderní stroje nabízejí, by přeci byla škoda.

V této práci se proto chci soustředit nejen na problematiky vývoje moderní webové aplikace, ale i na myšlenky a postupy za celým jejím designem. Při troše štěstí bude tento program ještě dlouhá léta dobře sloužit, než také podlehne digitálnímu stáří.