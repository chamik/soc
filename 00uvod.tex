\chapter*{Úvod}
\addcontentsline{toc}{chapter}{Úvod} % přidá položku úvod do obsahu

Každý žák Gymnázia Jana Palacha v Mělníku je povinen několikrát během svého studia podstoupit takzvaný \enquote{rozřazovák} -- test, který nás podle znalostí angličtiny rozdělí do několika jazykových skupin. Program, ve kterém jsme dříve pracovali, byl zastaralý už jen od pohledu a šel triviálně rozbít; každý rok se to někomu povedlo.

Program nakonec přestal fungovat úplně, rozhodl jsem se tedy testy přetvořit. Mým cílem bylo vytvořit webovou aplikaci, která bude generovat originální testy každému žákovi, dokáže spolehlivě otestovat desítky lidí naráz a výsledky automaticky zpracovat. Měla by používat nejmodernější volně dostupné technologie, být intuitivní a hlavně život žákům i učitelům usnadnit, ne naopak.

V práci se budu zabývat dvěma hlavními částmi programu; učitelskou, kde je možné vytvářet otázky, nastavovat testy a stahovat automaticky generované tabulky s výsledky, a dále pak žákovskou, kde jsou studenti testováni a jejich úspěšnost je vyhodnocena ihned po odevzdání.

V průběhu vývoje bylo potřeba myslet na několik klíčových problémů, zejména na správu uživatelů, spolehlivost testování a uživatelskou přívětivost.

Aplikace byla vytvořena na základě specifických požadavků vyučujících angličtiny na našem gymnáziu.