\chapter*{Úvod}
\addcontentsline{toc}{chapter}{Úvod} % přidá položku úvod do obsahu

Každý žák Gymnázia Jana Palacha na Mělníku je povinen několikrát během svého studia podstoupit takzvaný \enquote{rozřazovák} -- test, který nás podle znalostí angličtiny rozdělí do několika jazykových skupin. Program, ve kterém jsme dříve pracovali, byl zastaralý už jen od pohledu a šel triviálně rozbít; každý rok se to někomu povedlo.

Program nakonec přestal fungovat úplně. Od té doby se poslechová část dělá na papír a zbytek na počítači, což je nepohodlné nejen pro žáky, ale hlavně pro učitele. Rozhodl jsem se ho tedy přetvořit. Mým cílem je vytvořit program, který bude generovat originální testy každému žákovi, dokáže spolehlivě otestovat desítky lidí naráz a výsledky automaticky zpracovat. Měl by používat nejmodernější volně dostupné technologie, být intuitivní a hlavně život žákům i učitelům usnadnit, ne naopak.