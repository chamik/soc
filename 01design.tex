\hypertarget{Design}{\chapter{Design}}

\section{Motivace}

Při tvorbě aplikace bylo neustále nutné myslet na dva hlavní faktory, kolem kterých se celý design musí odvíjet; to sice, že je program mířený pro dvě skupiny: učitele a žáky zároveň.

Úkolem žáka je v testu odpovídat na otázky. Žák by měl mít co nejméně možností, jak podvádět, aby byl správně umístěn do jazykové skupiny úměrné jeho znalostem.

Učitel pracuje se svým vlastním rozhraním, kde zadává otázky, vytváří testy, kontroluje průběh testování a získává automaticky zpracované výsledky.

V obou případech je důležité, aby byla webová aplikace intuitivní na používání, měla příjemný design a hlavně spolehlivě fungovala.

\section{Přihlašování}

Ještě před tím, než se uživatelé dostanou do svého rozhraní, je potřeba, aby se přihlásili. Tato část aplikace je společná pro žáky i učitele. 

\textit{obrázek - Přihlašovací stránka je hlavní stránkou aplikace. }

Učitelé a žáci se přihlásí pomocí svého školního Google účtu (vizte kapitola 2.X.X).

Pokud se přihlásí učitel, zobrazí se mu nabídka pro přechod do administrátorské sekce.

\textit{obrázek - po přihlášení učitele}

Pokud se přilhásí žák, záleží na tom, zdali je pro jeho ročník aktuálně spuštěný test. Pokud ne, zobrazí se pouze informace o tom, že pro žáka žádný test aktuálně spuštěný není; pokud ano, zobrazí se informace o jeho trvání a počtu otázek, spolu s možností test spustit.

\textit{obrázek - po přihlášení žáka}

\section{Žákovské rozhraní}

Žákovské rozhraní slouží k vyplňování zadaných testů. Po přihlášení je přesměrován do rozhraní, kde jsou pro něj vygenerované otázky (vizte kapitola 2.X.X). Žák má na výběr ze čtyř možností z nihž je vždy pouze jedna správně.

\section{Učitelské rozhraní}