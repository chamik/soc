\hypertarget{Design}{\chapter{Design}}

\section{Motivace}

Při tvorbě aplikace bylo neustále nutné myslet na dva hlavní faktory, kolem kterých se celý design musí odvíjet; program je určený pro dvě skupiny: učitele a~žáky zároveň. 

Úkolem žáka je v časovém limitu odpovědět na jazykové otázky. Žák by měl mít co nejméně možností jak podvádět, aby byl správně umístěn do jazykové skupiny úměrné jeho znalostem; dostane náhodnou sadu otázek různých obtížností, které pro jeho ročník navolí učitel.

Učitel pracuje se svým vlastním rozhraním, kde zadává otázky, vytváří testy, \mbox{kontroluje} průběh testování a získává automaticky zpracované výsledky.

V obou případech je důležité, aby byla webová aplikace intuitivní na používání, měla příjemný design a hlavně spolehlivě fungovala.

\section{Přihlašování}
\label{sec:login-design}

Ještě před tím, než se uživatelé dostanou do svého rozhraní, je potřeba, aby se přihlásili. Tato část aplikace je společná pro žáky i učitele. 

\begin{figure}[H]
    \centering
    \includegraphics[width=350px]{images/01design/login.png}
    \caption{Přihlašovací stránka je hlavní stránkou aplikace.}
\end{figure}

Učitelé a žáci se přihlásí pomocí svého školního Google účtu (viz \ref{sec:login}).

\newpage

Pokud se přihlásí učitel, zobrazí se nabídka pro přechod do administrátorské sekce.

\begin{figure}[H]
    \centering
    \includegraphics[width=350px]{images/01design/teacher.png}
    \caption{Hlavní stránka po přihlášení učitele.}
\end{figure}

Pokud se přihlásí žák, ve výchozím stavu se mu zobrazí stránka s informací, že pro něj aktuálně není spuštěný žádný test.

\begin{figure}[H]
    \centering
    \includegraphics[width=350px]{images/01design/student-no-test.png}
    \caption{Hlavní stránka po přihlášení žáka bez zadaného testu.}
\end{figure}

Pokud je pro ročník přihlášeného žáka aktuálně aktivovaný test, je vidět jeho časové omezení, počet otázek a obtížnost. Pokud ne, zobrazí se pouze informace o tom, že pro žáka žádný test aktuálně spuštěný není.

\begin{figure}[H]
    \centering
    \includegraphics[width=350px]{images/01design/student-yes-test.png}
    \caption{Hlavní stránka po přihlášení žáka se zadaným testem.}
\end{figure}

Při aktivovaném testu může žák test spustit. Po spuštění se pro žáka vygeneruje originální sada otázek (viz \ref{userapi}) a žák je přesměrován do žákovského rozhraní. Pokud žák během testu tuto stránku opustí (například kvůli technickým potížím, výpadku proudu, atp.), zůstává test spuštěný a je možné pokračovat v jeho vyplňování z hlavní stránky. 

\begin{figure}[H]
    \centering
    \includegraphics[width=200px]{images/01design/continue.png}
    \caption{Při probíhajícím testu se text tlačítka změní na "Pokračovat" a zobrazí se pod ním zbývající čas.}
\end{figure}

Jakmile žák test odevzdá, je přesměrován zpět na hlavní stránku, kde mu je sdělen jeho výsledek.

\begin{figure}[H]
    \centering
    \includegraphics[width=350px]{images/01design/filled-out.png}
    \caption{Hlavní stránka s výsledky po odevzdání testu.}
\end{figure}

\section{Žákovské rozhraní}

Žákovské rozhraní slouží k vyplňování vygenerovaných testů. Pro každý ročník je předem nastavený čas a počet otázek různých obtížností (viz \ref{adminapi}). Z banky otázek je podle těchto kritérií několik náhodně vybráno a zobrazeno. Otázky mohou být různého druhu, ale u každé otázky je vždy na výběr ze čtyř možností, z nichž právě jedna je správná. V pravém horním rohu každé otázky lze slabě vidět ID otázky (pro usnadnění komunikace s učitelem, například pokud student v otázce najde chybu).

\begin{figure}[H]
    \centering
    \includegraphics[width=400px]{images/01design/otazky.png}
    \caption{Dvě vygenerované otázky. Tmavě fialové zbarvení značí výběr.}
    \label{purplerect}
\end{figure}

V pravém dolním rohu (na mobilních zařízeních na spodní straně) obrazovky se nachází informační box se zbývajícím časem, počtem zodpovězených otázek a celkovým počtem otázek.

\begin{figure}[H]
    \centering
    \includegraphics[width=200px]{images/01design/infobox.png}
    \caption{Informační box při vyplňování testu.}
\end{figure}

\begin{figure}[H]
    \centering
    \includegraphics[width=400px]{images/01design/submit.png}
    \caption{Otázka a tlačítko pro odevzdání testu.}
\end{figure}

Po vyplnění testu může žák test odevzdat stisknutím tlačítka. Pokud tak neučiní do konce časového limitu, test se odevzdá sám. Po odevzdání testu je uživatel přesměrován zpět na hlavní stránku.

Celé žákovské rozhraní je uzpůsobené i pro mobilní zařízení.

\begin{figure}[H]
    \centering
    \includegraphics[width=200px]{images/01design/test-mobile.png}
    \caption{Žákovské rozhraní na mobilním zařízení.}
\end{figure}

\section{Učitelské rozhraní}
\label{sec:admin}

Učitelské rozhraní je poměrně komplikované, proto je rozděleno do několika částí. V těchto částech se lze pohybovat pomocí takzvaného \enquote{navbaru}, který se nachází v~horní části obrazovky.

\begin{figure}[H]
    \centering
    \includegraphics[width=400px]{images/01design/navbar.png}
    \caption{Navbar v učitelském rozhraní.}
\end{figure}

Na jeho levé straně můžeme přepínat do různých částí programu; aktuálně vybranou část značí fialový obrys. Na pravé straně lze vidět uživatelské jméno a možnost odhlásit se.

Po přechodu do administrátorské sekce se učitel dostane do záložky \enquote{Přehled}, kde je vidět počet otázek v databance, aktuálně testovaných žáků a spuštěných testů. Níže se nachází jmenný seznam aktuálně testovaných.

\begin{figure}[H]
    \centering
    \includegraphics[width=400px]{images/01design/admin-index.png}
    \caption{Sekce Přehled.}
\end{figure}

\subsection{Testy}

Tato sekce programu slouží ke správě testů a ke stahování zpracovaných výsledků.

\begin{figure}[H]
    \centering
    \includegraphics[width=400px]{images/01design/tests.png}
    \caption{Sekce Testy.}
\end{figure}

Při prvotním spuštění programu se automaticky vygeneruje osm testů (pro osm ročníků) s identickým výchozím nastavením. 

\begin{figure}[H]
    \centering
    \includegraphics[width=400px]{images/01design/test.png}
    \caption{Detail komponentu testu.}
\end{figure}

V levé části se nachází číslo ročníku a status testu (\M{VYPNUTÝ}, \M{AKTIVNÍ} a~\M{VYPLNĚNÝ}). Ve sloupci Gramatika lze vidět počet použitých otázek pro každou jazykovou obtížnost. Ve sloupci Čas testu lze vidět časový limit v minutách. Úplně vpravo pak tlačítka pro nastavení a spuštění testu.

\begin{figure}[H]
    \centering
    \includegraphics[width=400px]{images/01design/test-modal.png}
    \caption{Modální okno nastavení testu.}
\end{figure}

V modálním\footnote{Modální okno je typ uživatelského rozhraní, které vynucuje interakci s uživatelem, dokud není dokončena nebo zrušena. Tato okna se často zobrazují před ostatním obsahem na obrazovce.} okně nastavení lze změnit hodnoty pro použití otázek pro každou jazykovou obtížnost, níže i čas na test. Všechny zadané informace jsou shrnuty ve větě, která se během zadávání dat živě aktualizuje.

Tlačítko \M{Uložit} je zde spíše pro efekt, protože se stejně vše po zavření okna automaticky uloží.

Po zmáčknutí tlačítka \M{Spustit} se test spustí, což je reflektováno ve statusu testu a~v~zablokování možnosti Nastavení. V tomto stavu máme pouze možnost test zastavit.

\begin{figure}[H]
    \centering
    \includegraphics[width=400px]{images/01design/test-running.png}
    \caption{Detail spuštěného testu.}
\end{figure}

Pokud jsou zadaná data testu nesprávná (například při zadání většího počtu otázek určité kategorie, než jich je v databance), zobrazí se učiteli informační okno.

\begin{figure}[H]
    \centering
    \includegraphics[width=400px]{images/01design/unable-to-run-test.png}
    \caption{Informační okno při nepodařeném spuštění testu.}
\end{figure}

Po zastavení testu se status změní na \M{VYPLNĚNÝ}. Z tohoto stavu lze test znovu spustit tlačítkem \M{Spustit} (toto je například potřeba, pokud během rozřazovacích testů někdo chyběl, což je poměrně pravděpodobné), \M{Smazat} výsledky (tato možnost test dostane zpět do stavu \M{VYPNUTÝ} a je nevratná) a nebo si stáhnout výsledky ve formátu \M{.xlsx} (Excel tabulka).

\begin{figure}[H]
    \centering
    \includegraphics[width=400px]{images/01design/test-filled.png}
    \caption{Detail vyplněného testu.}
\end{figure}

Při smazání výsledků testu je učitel důrazně varován, že je tato akce nevratná a že je potřeba stáhnout excel tabulku.

\begin{figure}[H]
    \centering
    \includegraphics[width=400px]{images/01design/test-deletion.png}
    \caption{Varování při smazání výsledků testu.}
\end{figure}

Výsledná tabulka má sloupce pro jméno, e-mail, procentuální úspěšnost, získané body a ID špatně zodpovězených otázek (viz program \ref{xslxgen}). Řádky jsou seřazeny sestupně dle úspěšnosti.

\begin{figure}[H]
    \centering
    \includegraphics[width=400px]{images/01design/excel.png}
    \caption{Ukázka tabulky s výsledky.}
\end{figure}

\pagebreak
\subsection{Gramatika}

Sekce Gramatika slouží k upravování gramatických otázek. Skládá se ze seznamu uložených otázek v databázi. V rychlém náhledu můžeme vidět text otázky, správnou odpověď, jazykovou úroveň a ID. V horním pravém rohu této sekce se nachází tlačítko pro vytvoření nové otázky. Stejné tlačítko se nachází i na konci seznamu.

\begin{figure}[H]
    \centering
    \includegraphics[width=400px]{images/01design/grammar.png}
    \caption{Sekce Gramatika.}
\end{figure}

Po kliknutí na otázku se otevře modální okno s nastavením dané otázky. Při kliknutí na tlačítko pro vytvoření otázky se otevře stejné okno, akorát nevyplněné. Zde je potřeba vyplnit text otázky, správnou odpověď, tři špatné odpovědi a vybrat jednu z jazykových obtížností. 

\begin{figure}[H]
    \centering
    \includegraphics[width=400px]{images/01design/question.png}
    \caption{Modální okno nastavení gramatické otázky.}
    \label{questionfill}
\end{figure}

Program automaticky kontroluje správnost zadaných údajů; zdali se nějaká ze špatných odpovědí neshoduje se správnou a zdali jsou všechna políčka vyplněná. Výchozí hodnota jazykové obtížnosti je A1.

\begin{figure}[H]
    \centering
    \includegraphics[width=400px]{images/01design/question-bad.png}
    \caption{Modální okno nastavení gramatické otázky s nesprávně vyplněnými hodnotami.}
\end{figure}

Pokud nejsou všechna pole vyplněna správně, nelze otázku uložit. Stále je však možné otázku smazat kliknutím na ikonku odpadkového koše v pravém horním rohu modálního okna.

\pagebreak
\subsection{Zálohy}

Tato sekce slouží ke správě záloh databanky otázek a nastavení testů. 

\begin{figure}[H]
    \centering
    \includegraphics[width=400px]{images/01design/backup.png}
    \caption{Sekce Zálohy.}
\end{figure}

Zde je možné stáhnout a nahrát soubor zálohy, který obsahuje všechny gramatické otázky a nastavení testů. Nahrání zálohy nenávratně nahradí nastavení v databázi, tudíž je učitel před přepsáním varován.

\begin{figure}[H]
    \centering
    \includegraphics[width=400px]{images/01design/backup-warning.png}
    \caption{Varování při nahrávání zálohy.}
\end{figure}

Tato stránka také obsahuje krátký informační text ohledně rozsahu záloh a dobrých praktik zálohování dat.

\pagebreak
\section{O programu}

Na hlavní stránce se na spodní straně obrazovky nachází odkaz na informace \M{O~programu}. Nachází se zde stručný popis toho, o čem program je, kde lze nalézt zdroj programu a kdo je jeho aktuální správce (nastavitelné v konfiguraci, viz \ref{sec:config}).

\begin{figure}[H]
    \centering
    \includegraphics[width=400px]{images/01design/about.png}
    \caption{Stránka O programu.}
\end{figure}
