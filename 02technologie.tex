\hypertarget{Technologie}{\chapter{Technologie}}

\section{Požadavky}

Hlavní podmínkou pro použití technologie v této odborné práci je její volná licence. Dalšími podmínkami byla jednoduchost používání/naučení a aktuálnost -- cílem je použít co nejmodernější nástroje, aby program stárnul co nejpomaleji.

\section{Licence}

Důležitým uvážením při tvorbě projektu byl výběr licence. Nakonec jsem se rozhodl pro licenci MIT, která je krátká, jednoduchá na pochopení, umožňuje s kódem volně nakládat a nezaručuje z mé strany žádnou zodpovědnost a záruku.\cite{choosealicense}

Volba licence je důležitá, protože\dots

\section{Stack}

Dnes existuje nepřeberně způsobů, jak vytvořit full-stack\footnote{Full-stack označuje kompletní řešení, často za použití serverové i klientské aplikace.} aplikaci a vybrat mezi nimi není jednoduché. Mimo to je tyto technologie potřeba často kombinovat; této kombinaci se nazývá stack. Stack určuje, jak aplikaci vyvíjíme a jaké prostředky nám jsou dostupné. Stack si lze vytvořit sám dle vlastního uvážení a potřeb, nebo využít nějaký volně dostupný, který už je ozkoušený a důvěryhodný.

Mojí volbou se stal \M{T3 Stack}, který kombinuje několik populárních technologií, které jsou blíže popsány níže. Zároveň je však poměrně tvárný a lze ho jednoduše přizpůsobit potřebám projektu.\cite{t3stack}

\subsection{Jazyk}

Jedno z prvních rozhodnutí, které musí člověk při tvorbě projektu udělat je samotný výběr programovacího jazyka. Posledních pár let se většina vývoje soustředí kolem jazyka \M{TypeScript}, který je typovanou\footnote{Typy v programovacím jazyce definují struktury, ve kterých se mimo jiné dají ukládat data. Pokud předem víme, jak tyto struktury vypadají, dá se jednodušeji při programování vyhnout chybám.} variantou jazyka \M{JavaScript}. \M{T3 Stack} ho automaticky používá. 

\subsection{Framework}

Framework je jakási nadstavba nad programovacím jazykem a má proces vytváření webových aplikací usnadnit. Při tvorbě webové aplikace fungují jako páteř, na které stojí vše ostatní. \M{T3 Stack} přichází s frameworkem \M{Next.js}, který je sám o sobě nadstavbou nad populárním frameworkem \M{React}. Nabízí mnoho funkcí, jmenovitě například možnost výběru stylu renderování stránky, automatické optimalizace na několika frontách, hot code reloading atp.\cite{nextjs}

\subsection{Databáze}

Databáze je klíčovou součástí jakékoli větší aplikace. Pro tento projekt jsem zvolil \M{PostgreSQL}; jedná se o jeden z nejvšestranějších volně dostupných databázových systémů.

\M{T3 Stack} přichází s knihovnou \M{Prisma}, která komunikaci s databázi usnadňuje například tím, že pro TypeScript generuje typy podle definovaného schéma.

\subsection{CSS Framework}

Na rozdíl od jiných populárních CSS frameworků, jako je například \M{Bootstrap} nebo \M{Skeleton} se \M{TailwindCSS} liší tím, že nenabízí již předem připravené komponenty, ale pouze profesionály definované CSS třídy. Vývojář díky tomu není žádným způsobem omezován.\footnote{Navíc, pokud je člověk ve tvorbě webů zběhlý, na první pohled dokáže poznat připravené komponenty z populárních CSS frameworků, což je vzhled, kterému jsem se chtěl vyvarovat.} \M{TailwindCSS} je automaticky obsažen v \M{T3 Stack}u.

\subsection{Autentifikace}

Autentifikace je při tvorbě webových aplikací častým úskalím, proto \M{T3 Stack} automaticky přichází s \M{NextAuth.js}, což je rozšíření pro použitý framework \M{Next.js}. Umožňuje přihlašování s Google účtem, což je pro tuto aplikaci potřeba. Jedná se o léta používaný standard, což je u což je u bezpečnostně kritického komponentu stacku potřebné.

\section{Hosting}

Jakmile je program hotový, musí někde běžet \dots