\hypertarget{Technologie}{\chapter{Technologie}}

\section{Požadavky}

Hlavní podmínkou pro použití technologie v této odborné práci je její volná licence. Dalšími podmínkami byla jednoduchost používání/naučení a modernost.

\section{Licence}

Dalším důležitým uvážením při tvorbě projektu byl výběr licence. Nakonec jsem se rozhodl pro licenci MIT, která je krátká, jednoduchá na pochopení, umožňuje s kódem volně nakládat a nezaručuje z mé strany žádnou zodpovědnost a záruku.\cite{choosealicense}

Volba licence je důležitá, protože\dots

\section{Stack}

Dnes existuje nepřeberně způsobů, jak vytvořit full-stack\footnote{Full-stack označuje kompletní řešení, často za použití serverové i klientské aplikace.} aplikaci a vybrat mezi nimi není jednoduché. Mimo to je tyto technologie potřeba často kombinovat; této kombinaci se nazývá stack. Stack určuje, jak aplikaci vyvíjíme a jaké prostředky nám jsou dostupné. Stack si lze vytvořit sám dle vlastního uvážení a potřeb, nebo využít nějaký volně dostupný, který už je ozkoušený a důvěryhodný.

Mojí volbou se stal \M{T3 Stack}, který kombinuje několik populárních technologií, které jsou blíže popsány níže. Zároveň je však poměrně tvárný a lze ho jednoduše přizpůsobit potřebám projektu.

\subsection{Jazyk}

Jedno z prvních rozhodnutí, které musí člověk při tvorbě projektu udělat je samotný vývěr programovacího jazyka. Posledních pár let se většina vývoje soustředí kolem jazyka \M{TypeScript}, který je typovanou\footnote{Typy v programovacím jazyce definují struktury, ve kterých se mimo jiné dají ukládat data. Pokud předem víme, jak tyto struktury vypadají, dá se jednodušeji při programování vyhnout chybám.} variantou jazyka \M{JavaScript}. \M{T3 Stack} ho automaticky používá. 

\subsection{Framework}

Framework je jakási nadstavba nad programovacím jazykem a má proces vytváření webových aplikací usnadnit. Při tvorbě webové aplikace fungují jako páteř, na které stojí vše ostatní. \M{T3 Stack} přichází s frameworkem \M{Next.js}, který je sám o sobě nadstavbou nad populární knihovnou \M{React}. Nabízí mnoho funkcí, jmenovitě například možnost výběru stylu renderování stránky, nebo automatické optimalizace na několika frontách.

\subsection{Databáze}

Databáze je klíčovou součástí jakékoli větší aplikace. \M{T3 Stack} přichází s knihovnou \M{Prisma}, která komunikaci s ní v hodně ohledech usnadňuje. Co se samotné databáze týče, zvolil jsem \M{PostgreSQL}, která je léty osvědčená a jedna z nejvšestranějších volně dostupných databázových systémů.